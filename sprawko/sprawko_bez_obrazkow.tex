\documentclass{classrep}
\usepackage[utf8]{inputenc}
\usepackage[pdftex]{graphicx}
\usepackage[polish]{babel}
\usepackage{algorithm}
\usepackage{algorithmic}
\usepackage{multicol}
\usepackage{amsmath}
\usepackage{listings}
\usepackage{array}
\usepackage{multirow}
\usepackage{url}
\usepackage{subfigure}
\usepackage{hyperref}

\studycycle{Informatyka, studia dzienne, II st.}
\coursesemester{I}

\coursename{Przetwarzanie obrazów i dźwięków}
\courseyear{2010/2011}

\courseteacher{dr inż. Bartłomiej Stasiak}
\coursegroup{micmic}
\svnurl{https://serce.ics.p.lodz.pl/svn/labs/poid/bsat_wt1415/micmic}

\author{%
  \studentinfo{Michał Janiszewski}{169485} \and
  \studentinfo{Michał Kawski}{169487}
}

\title{Zadanie 1: Przetwarzanie obrazów}

\floatname{algorithm}{Algorytm}

\usepackage{wasysym}

\begin{document}

\maketitle

\section{Cel zadania}
Celem zadania było stworzenie szkieletu aplikacji do przetwarzania i analizy obrazów i dźwięków oraz zaimplementowanie następujących operacji dotyczących przetwarzania obrazu:
\begin{itemize}
 \item operacje podstawowe:
  \begin{itemize}
   \item zmiana kontrastu obrazu,
   \item zmiana jasności obrazu,
   \item wyznaczanie negatywu obrazu,
  \end{itemize}
 \item filtrowanie obrazu w oparciu o:
  \begin{itemize}
   \item filtr ze średnią arytmetyczną,
   \item filtr medianowy,
  \end{itemize}
 \item wyznaczanie i pokazywanie histogramu obrazu dla poszczególnych jego kanałów, a także umożliwienie edycji obrazu na podstawie tego histogramu,
 \item filtracja liniowa oparta o splot z możliwością zdefiniowania rozmiaru i~wartości maski,
 \item filtracja nieliniowa w dziedzinie czasu.
\end{itemize}

Wylosowane warianty zadań to:
\begin{itemize}
 \item histogramy H3 \ppauza wyjściowa gęstość prawdopodobieństwa podana wzorem Raleigha:
  \begin{equation*}
   g(f) = g_{min} + \left(2 \alpha^2 \ln \left(\frac{1}{N} \displaystyle\sum\limits_{m=0}^f H\left(m\right)\right)^{-1}\right)^{\frac{1}{2}}
  \end{equation*}
 \item splot S4 \ppauza wydobywanie szczegółów z tła: południe, południowy-zachód, zachód, północny-zachód:
\begin{equation*}
\left(\begin{array}{ccc} -1 & -1 & -1 \\
1 & -2 & 1 \\
1 & 1 & 1 \\
\end{array}\right)\\
\\
\left(\begin{array}{ccc} 1 & -1 & -1 \\
1 & -2 & -1 \\
1 & 1 & 1 \\
\end{array}\right)\\
\\
\left(\begin{array}{ccc} 1 & 1 & -1 \\
1 & -2 & -1 \\
1 & 1 & -1 \\
\end{array}\right)\\
\\
\left(\begin{array}{ccc} 1 & 1 & 1 \\
1 & -2 & -1 \\
1 & -1 & -1 \\
\end{array}\right)
\end{equation*}
  \item filtracja nieliniowa O5 \ppauza operator Rosenfelda:
\begin{equation*}
 g(x, y) = \frac{1}{R} \left( \displaystyle \sum \limits_{i=1}^R f\left(x + i - 1, y\right) - \displaystyle \sum \limits_{i=1}^R f \left(x - i, y\right) \right)
\end{equation*}
\end{itemize}

\section{Opis metod przetwarzania}

Poniżej znajduje się opis algorytmów realizujących opisane powyżej zagadnienia. Obsługiwane są 3 typy obrazów:
\begin{enumerate}
 \item obrazy kolorowe (24-bitowe),
 \item obrazy w 8-bitowej skali szarości,
 \item obrazy binarne (1-bitowe).
\end{enumerate}

W przypadku obrazu kolorowego zakładamy, że każdy piksel ma trzy kanały: R (czerwony), G (zielony) i B (niebieski), a wartość każdego z tych pikseli znajduje się w zakresie $[0; 255]$ i przyjmuje tylko wartości całkowite.

W przypadku obrazów w skali szarości przyjmujemy, że każdy piksel posiada tylko jedną, 8-bitową składową, jest to indeks koloru z palety barw zapisanej w obrazie. Chociaż możliwe jest zapisanie dowolnej palety kolorów, przyjmujemy, że mamy do czynienia tylko i wyłącznie z obrazami w~skali szarości (lub innego koloru), a 256 dostępnych, unikatowych kolorów w palecie uporządkowanych jest od najjaśniejszego (białego) do najbardziej intensywnego (czarnego \ppauza w przypadku skali szarości).

Piksele w obrazach binarnych przyjmują tylko dwa dostępne stany: biały i czarny.

Ponadto, przyjmijmy następujące oznaczenia:
\begin{description}
 \item[$g(n, m)$] wartość kanału piksela o współrzędnych $(n, m)$,
 \item[$g'(n, m)$] piksel o współrzędnych $(n, m)$ w obrazie wyjściowym,
 \item[$N, M$] odpowiednio szerokość i wysokość obrazu w pikselach.
\end{description}


\subsection{Operacje podstawowe}

Opisane poniżej operacje podstawowe ograniczają swoje działanie tylko do jednego piksela, tzn. wartość funkcji przekształcenia nie zależy od otoczenia w jakim piksel się znajduje.

Algorytmy abstrachują od szczegółów implementacji, dzięki czemu można opisać ich działanie dla wszystkich wymienionych wyżej typów obrazów.

Ponieważ piksele w operacjach tych przetwarzane są niezależnie, przetwarzanie obrazu za ich pomocą sprowadza się do przetworzenia każdego piksela obrazu zadanym algorytmem.

\subsubsection{Negatyw}
\label{filter.negative}
Wyznaczanie negatywu obrazu sprowadza się do wyznacznia wartości przeciwnej dla każdego z kanałów piksela:

\begin{equation}
 g'(n, m) = -g(n, m) \mod C_{max}
\end{equation}

gdzie $C_{max}$ to maksymalna możliwa wartość.

\subsubsection{Zmiana kontrastu}
\label{filter.contrast}
Zmiana kontrastu to zmiana amplitudy charakteryzującej każdy z kanałów piksela. Dokonujemy tego poprzez mnożenie wartości każdego z kanałów przez pewien współczynnik $\alpha \in [0; 2]$, a następnie ograniczamy tak uzyskane wartości do zakresu możliwych:
\begin{equation}
\label{eq.contrast}
 f(n, m) = g(n, m) \cdot \alpha
\end{equation}
\begin{equation}
 g'(n, m) = \left\{
  \begin{array}{l l}
    0 & \quad f(n, m) < 0 \\
    f(n, m) & \quad 0 \leq f(n, m) \leq C_{max} \\
    C_{max} & \quad f(n, m) > C_{max} \\
  \end{array}
\right.
\end{equation}

\subsubsection{Zmiana jasności}
\label{filter.brightness}
Zmiana jasności sprowadza się do dodania pewnej stałej $\delta$ do każdego kanału piksela. $\delta$ może przyjmować wartości zarówno dodatnie jak i ujemne, co określa odpowiednio rozjaśnianie oraz przyciemnianie obrazu. Tak przetworzone wartości należy ograniczyć jeszcze do wartości możliwych do zapisania w obrazie:
\begin{equation}
\label{eq.brightness}
 f(n, m) = g(n, m) + \delta
\end{equation}
\begin{equation}
 g'(n, m) = \left\{
  \begin{array}{l l}
    0 & \quad f(n, m) < 0 \\
    f(n, m) & \quad 0 \leq f(n, m) \leq C_{max} \\
    C_{max} & \quad f(n, m) > C_{max} \\
  \end{array}
\right.
\end{equation}

\subsection{Filtrowanie}
Filtrowanie obrazu jest próbą poprawienia jego jakości \ppauza zarówno obiektywnej jak i subiektywnej. Zadaniem zaimplementowanych filtrów jest usuwanie szumów z obrazu.

Każdy z tych filtrów, w odróżnieniu od operacji podstawowych, uwzględnia okolicę piksela, do której należy przetwarzany aktualnie piksel. Wielkość otoczenia nazywana jest wielkością maski i określa, ile pikseli zarówno w poziomie jak i pionie branych jest pod uwagę podczas wyliczania nowej wartości przetwarzanego piksela.

W przypadku gdy maska określa pobranie wartości piksela spoza obrazka, element taki zostaje zignorowany i nie jest brany pod uwagę w obliczeniach.

\subsubsection{Filtr ze średnią arytmetyczną}
\label{filter.average}
Dla każdego kanału piksela obrazu sumowane są wartości tego kanału z wszystkich pikseli odległych od aktualnie przetwarzanego nie więcej, niż określa to rozmiar maski. Suma ta jest następnie dzielona przez ilość uwzględnionych elementów \ppauza obliczana jest średnia arytmetyczna:
\begin{equation}
 g'(n, m) = \frac{1}{(2 \cdot w + 1) \cdot (2 \cdot h + 1)} \cdot \displaystyle \sum \limits_{n' = n - w}^{n + w} \displaystyle \sum \limits_{m' = m - h}^{m + h} g(n', m')
\end{equation}
gdzie $w$ to promień poziomy, a $h$ promień pionowy maski.

\subsubsection{Filtr medianowy}
\label{filter.median}
W przypadku filtru medianowego wartości kanału ze wszystkich pikseli wchodzących w skład maski ustawiane są w niemalejącym szeregu, a następnie zwracana jest mediana tych elementów, czyli element z pozycji $\frac{l}{2}$~w~przypadku szeregu o nieparzystej długości $l$ lub średnia arytmetyczna z~dwóch środkowych elementów $\frac{g(\frac{l}{2} - 1) + g(\frac{l}{2})}{2}$ w przypadku szeregu o parzystej ilości elementów.

\subsection{Modyfikacja w oparciu o histogram}
Przetwarzanie obrazu w oparciu o histogram, wykorzystująca wzór Raleigha wymaga podania minimalnej wartości koloru piksela oraz parmetru $\alpha$. 
Edycja obrazu dyktowana jest poniższym wzorem:
\begin{equation}
g(f) = g_{min} + (2\alpha^{2}ln(\frac{1}{N} \displaystyle \sum \limits_{m=0}^{f} H(m))^{-1})^{\frac{1}{2}}
\end{equation}

\subsection{Filtracja liniowa w oparciu o splot}
\label{filter.convolution}
Aby dokonać filtracji liniowej opartej o splot, konieczna jest znajomość maski filtru, jaka opisuje odpowiedź impulsową splotu. Nazwijmy tę maskę macierzą $X[2 \cdot W + 1; 2 \cdot H + 1]$ i przyjmijmy, że funkcja $x(w, h)$ zwraca element macierzy $X$ z pozycji $(w, h)$, wtedy:
\begin{equation}
 g'(n, m) = \displaystyle \sum \limits_{w=-W}^W \displaystyle \sum \limits_{h=-H}^H g(n + w, m + h) \cdot x(w, h)
\end{equation}

Uzyskane w taki sposób wartości należy znormalizować. W tym celu postanowiliśmy znaleźć minimalną i maksymalną wartość kanału $C'_{min}$ oraz $C'_{max}$ po przekształceniu, a następnie przeskalować wszystkie wartości z tego zakresu na zakres możliwy do osiągnięcia $C_{min} \div C_{max}$\footnote{Wartość $C_{min}$ wynosi 0, zatem można ją pominąć w obliczeniach}:
\begin{equation}
 \label{eq:normalization}
 c = \frac{C_{max}}{C'_{max} - C'_{min}} \cdot (c' - C'_{min})
\end{equation}

\subsection{Filtracja nieliniowa}
\label{filter.nonlinear}
Wariant zadania, który nam przypadł przewidywał implementację poziomego operatora Rosenfelda:
\begin{equation}
 g'(n, m) = \frac{1}{R} \cdot \left( \displaystyle \sum \limits_{i=1}^R g\left(n + i - 1, m\right) - \displaystyle \sum \limits_{i=1}^R g\left(n - i, m\right) \right)
\end{equation}
Normalizacja uzyskanych danych ma postać jak w równaniu \ref{eq:normalization}.

\subsection{Obiektywna ocena jakości obrazu}
Ponieważ do każdego z zaszumionych obrazów testowych dostępne są wersje oryginalne, o idealnej jakości, można dokonać obiektywnej oceny jakości filtrowania. Zaimplementowane metody porównują odstępstwa przetworzonego obrazu zaszumionego z obrazem oryginalnym.

\subsubsection{MSE}
MSE, czyli \textit{Mean Square Error}, błąd średniokwadratowy $E$, jest podstawowym kryterium pomiarowym. Zlicza on sumę kwadratów różnic pomiędzy wartościami otrzymanymi, a idealnymi:
\begin{equation}
 E = \frac{1}{N \cdot M} \displaystyle \sum \limits_{n=0}^{N-1} \sum \limits_{m=0}^{M-1} \left(g\left(n, m\right) - g'\left(n, m\right) \right)^2
\end{equation}
W tym przypadku $g'$ odnosi się do przefiltrowanego obrazu.

\subsubsection{SNR}
SNR, czyli \textit{Signal to Noise Ratio}, współczynnik sygnału do szumu $R$, stanowi często spotykaną metodę określania jakości sygnału. Wyznaczany jest wzorem:
\begin{equation}
 R = 10 \log_{10} \frac{\displaystyle \sum \limits_{n=0}^{N-1} \sum \limits_{m=0}^{M-1} \left(g\left(n, m\right)\right)^2}{E}
\end{equation}
$E$ to wyznaczony powyżej współczynnik MSE.

\clearpage

\section{Implementacja}
Program powstał przy wykorzystaniu frameworku Qt wykorzystującego język C++.

Po uruchomieniu programu prezentowane jest użytkownikowi okienko klasy \texttt{MainWindow}. Umożliwia ono wybranie pliku z obrazem do otwarcia. Użytkownik może wybrać dowolny format wspierany przez bibliotekę Qt, a dzięki możliwości pisania wtyczek \ppauza zakres ten można powiększać.

Po wybraniu pliku obrazu do otwarcia, tworzona jest nowa instancja klasy \texttt{PhotoWindow}. W konstruktorze ładuje on podany jako argument plik, ustawia go jako prezentowany za pomocą kontrolki klasy \texttt{QLabel} obrazek oraz inicjalizuje filtry.

Każdy z filtrów dziedziczy po interfejsie \texttt{FilterInterface} skonstruowanym za pomocą klasy wirtualnej. Część metod tej klasy jest zaimplementowana \ppauza dostarczają one środków do identyfikacji filtrów. Do implementacji w~klasie pochodnej pozostawione zostały metody:
\begin{itemize}
 \item \texttt{virtual QString name() const} \ppauza zwracająca nazwę filtru,
 \item \texttt{virtual QImage apply()} \ppauza uruchamiająca działanie filtru.
\end{itemize}

Ponadto filtry powinny też dostarczać implementacji metody \texttt{virtual bool setup(const QImage \&)} w której inicjalizowane są parametry filtru. Podejście takie pozwala sprawdzić, czy użytkownik zatwierdził wykonanie operacji oraz czy wprowadzone parametry są poprawne. Dopiero po tych testach może zostać zwrócona wartość \texttt{true}, która komunikuje gotowość do uruchomienia filtra.

Z menu użytkownik może wybrać także opcję zapisania obrazka \ppauza obrazek zapisywany jest w formacie PNG, pokazania lub ukrycia okna histogramów, uruchomienia wspomnianych filtrów oraz sprawdzenia jakości obrazka.

W tym ostatnim przypadku należy załadować do programu zaszumiony obrazek i poddać go litracji, po kliknięciu wspomnianej opcji pojawi się okienko, w którym użytkownik powinien wskazać położenie pliku oryginalnego, zaś parametry MSE i SNR zostaną wypisane na wyjście błędów.

Uruchomienie filtru na obrazku skutkuje utworzeniem nowego okna klasy \texttt{PhotoWindow} wyświetlającym przetworzony obrazek. Tak wyświetlone okno w swoim tytule pokazuje, jakie operacje były wykonane na obrazku.

\subsection{Filtry}
Wszystkie filtry, w tym splot oraz filtracja nieliniowa opierają się o klasę \texttt{FilterInterface}. Instancje klas dziedziczących po niej tworzone są w konstruktorze \texttt{PhotoWindow} i wypełniane jest nimi submenu Filters. Wywołanie jednej z dostępnych opcji skutkuje wywołaniem metody \texttt{setup()}, a jeśli zwróci ona wartość true, wykonywana jest także metoda \texttt{apply}, która zwraca przetworzony obrazek.

\subsubsection{Negatyw - Negative}
Filtr ten nie posiada opcji, wykonuje metodę opisaną w sekcji \ref{filter.negative}.

\subsubsection{Jasność - Brightness}
Po wybraniu tego filtru prezentowane jest użytkownikowi okienko, w którym określany jest parametr $\delta$ opisany w sekcji \ref{filter.brightness}, a następnie metoda z~tej sekcji wykonywana jest dla całego obrazu.

\subsubsection{Kontrast - Contrast}
Wybranie tego filtru skutkuje prezentacją użytkownikowi okienka z suwakiem służącym do wyboru pożądanej wielkości parametru $\alpha$ z sekcji \ref{filter.contrast}. Możliwe jest wybranie wartości z zakresu $-255 \div 255$, wybrana wartość jest następnie dzielona przez 255 i dodawana do 1, aby przeskalować ją na zakres $[0; 2]$.

\subsubsection{Uśrednianie - Average}
Uruchomienie tego filtru powoduje pokazanie okienka z wyborem wielkości maski \ppauza użytkownik powinien wprowadzić promień poziommy i pionowy maski, po czym następuje wykonanie metody opisanej w \ref{filter.average}.

\subsubsection{Mediana - Median}
Włączenie tego filtru spowoduje działanie podobne jak w powyższym przypadku \ppauza ukazanie się okienka z wyborem maski, a następnie zostanie wykonana metoda z sekcji \ref{filter.median}.

\subsubsection{Filtr liniowy - Convolution}
Wybranie tej opcji spowoduje najpierw pokazanie okienka z wyborem wielkości maski, następnie zaś użytkownik zostanie poproszony o wprowadzenie wartości maski. Metoda ta jest opisana w sekcji \ref{filter.convolution}.

\subsubsection{Filtr nieliniowy - Rosenfeld}
Opcja ta spowoduje pokazanie okienka z suwakiem, w którym użytkownik powinien podać parametr $R$.

\subsubsection{Edycja za pomocą histogramu}
Edycja histogramu powoduje utworzenie kopii obrazu a następnie zmianę wartości barwy poszczególnych pikseli według wzoru umieszczonego w punkcie „Modyfikacja w oparciu o histogram”. Po wykonaniu tej operacji zostaje utworzony nowy obiekt klasy PhotoWindow, który prezentuje przetworzony obraz. Obraz  ten może być dalej przetwarzany na takiej samej zasadzie, co obraz otwarty z pliku.

\clearpage

\section{Materiały i metody}
Do testów wykorzystaliśmy popularny obrazek ,,Lena'' w wariantach kolorowym i odcieni szarości. Obrazki te prezentują rysunki \ref{fig.lenac} i \ref{fig.lena}.

Ponadto dla każdego wariantu wybraliśmy trzy rodzaje szumów:
\begin{itemize}
 \item szum o rozkładzie normalnym (rysunki \ref{fig.lenac_normal3} i \ref{fig.lena_normal3}),
 \item szum o rozkładzie jednostajnym (rysunki \ref{fig.lenac_uniform3} i \ref{fig.lena_uniform3}),
 \item szum o rozkładzie impulsowym (rysunki \ref{fig.lenac_impulse3} i \ref{fig.lena_impulse3}).
\end{itemize}

Podstawowe wersje rysunków zostały poddane działaniu wszystkich filtrów, zaś obrazy zaszumione tylko filtrowaniu z~wykorzystaniem średniej i~mediany.

Poniżej znajdują się opisane szczegółowo działania.

\subsection{Negatyw}
\label{sec.tests.negative}
Wyznaczony został negatyw obrazków za pomocą filtru Negative.

\subsection{Zmiana jasności}
\label{sec.tests.brightness}
Dokonano zmiany jasności obrazków za pomocą filtru Brightness. Obrazki przyciemniono oraz rozjaśniono stosując wartość parametru wynoszącą odpowiednio -85 oraz +85, co odpowiada wartościom $-1\frac{1}{3}$ oraz $1\frac{1}{3}$ parametru $\delta$ ze wzoru \ref{eq.brightness}.

\subsection{Zmiana kontrastu}
\label{sec.tests.contrast}
Dokonano zmiany kontrastu obrazków za pomocą filtru Contrast. Obrazkom zwiększono i zmniejszono kontrast stosując do tego następujące wartości parametru $\alpha$: -85 oraz +85 ze wzoru \ref{eq.contrast}.

\subsection{Filtrowanie}
\label{sec.tests.filtering}
\subsubsection{Za pomocą średniej arytmetycznej}
\label{sec.tests.filtering.average}
Dokonano trzech filtrowań, z maskami o wielkościach $3 \times 3$, $5 \times 5$, $7 \times 7$ za pomocą filtru Average. Filtrowaniu poddane zostały obrazki oryginalne oraz zaszumione.

\subsubsection{Za pomocą mediany}
\label{sec.tests.filtering.median}
Dokonano trzech filtrowań, z maskami o wielkościach $3 \times 3$, $5 \times 5$, $7 \times 7$ za pomocą filtru Median. Filtrowaniu poddane zostały obrazki oryginalne oraz zaszumione.

\subsection{Modyfikacja w oparciu o histogram}
\label{sec.tests.histogram}
Modyfikacja odbywa się w odniesieniu do trzech kolorów składowych - czerwonego, zielonego, niebieskiego oraz koloru szarości. W przykładzie edytowana została częśc histogramu dotycząca barwy czerwonej.

\subsection{Filtracja liniowa w oparciu o splot}
\label{sec.tests.convolution}
Dokonano filtracji liniowej opartej o splot. Przydzielony wariant zadania określał wykorzystanie następujących masek: wydobywanie szczegółów z tła: południe, południowy-zachód, zachód, północny-zachód:
\begin{equation*}
\left(\begin{array}{ccc} -1 & -1 & -1 \\
1 & -2 & 1 \\
1 & 1 & 1 \\
\end{array}\right)\\
\\
\left(\begin{array}{ccc} 1 & -1 & -1 \\
1 & -2 & -1 \\
1 & 1 & 1 \\
\end{array}\right)\\
\\
\left(\begin{array}{ccc} 1 & 1 & -1 \\
1 & -2 & -1 \\
1 & 1 & -1 \\
\end{array}\right)\\
\\
\left(\begin{array}{ccc} 1 & 1 & 1 \\
1 & -2 & -1 \\
1 & -1 & -1 \\
\end{array}\right)
\end{equation*}

\subsection{Filtracja nieliniowa}
\label{sec.tests.filtering.nonlinear}
Dokonano filtracji nieliniowej obrazu wykorzystując w tym celu operator Rosenfelda z następującymi parametrami R: 2, 3, 4, 5, 6.

\subsection{Obiektywna i subiektywna ocena jakości obrazu}
\label{sec.tests.quality}
Przefiltrowane zaszumione wersje obrazów poddano ocenie. W celu obiektywnej oceny jakości wykorzystano wskaźniki MSE i SNR, natomiast do oceny subiektywnej wykorzystano własne oczy \smiley.

\clearpage

\section{Wyniki}
Poniżej przedstawione są wyniki operacji przeprowadzonych na obrazkach.

\subsection{Negatyw}
Obrazki \ref{fig.lenac_negative} i \ref{fig.lena_negative} przedstawiają negatywy obrazków, jak opisano w~sekcji \ref{sec.tests.negative}.

\subsection{Zmiana jasności}
Obrazki \ref{fig.lenac_brightness_+85}, \ref{fig.lenac_brightness_-85}, \ref{fig.lena_brightness_+85} i \ref{fig.lena_brightness_-85} prezentują obrazki ze zmodyfikowaną jasnością zgodnie z opisem z \ref{sec.tests.brightness}.

\subsection{Zmiana kontrastu}
Obrazki \ref{fig.lenac_contrast_+85}, \ref{fig.lenac_contrast_-85}, \ref{fig.lena_contrast_+85} i \ref{fig.lena_contrast_-85} prezentują obrazki ze zmodyfikowaną jasnością zgodnie z opisem z \ref{sec.tests.contrast}.

\subsection{Filtrowanie}
Do tego filtru wykorzystano zarówno obrazki oryginalne jak i zaszumione. Modyfikacja polegała na przefiltrowaniu obrazka wybraną metodą z następującymi wielkościami masek: $3 \times 3$, $5 \times 5$ oraz $7 \times 7$.

\subsubsection{Za pomocą średniej arytmetycznej}
Wyniki wykonanych operacji prezentują obrazki \ref{fig.lenac_average}, \ref{fig.lena_average}, \ref{fig.lenac_normal3_average}, \ref{fig.lenac_uniform3_average}, \ref{fig.lenac_impulse3_average}, \ref{fig.lena_normal3_average}, \ref{fig.lena_uniform3_average} i \ref{fig.lenac_impulse3_average}.

%%%%%%%%%%%%%%%%%%%%%%%%%%%%%%%%%%%%%%%%%%%%%%%%%%%%%%%%%%%%%%%%%%%%%%%%%%%%%%%%%%%%%%%%%%%%%%%%%%%
%%%%%%%%% average - original

%%%%%%%%%%%%%%%%%%%%%%%%%%%%%%%%%%%%%%%%%%%%%%%%%%%%%%%%%%%%%%%%%%%%%%%%%%%%%%%%%%%%%%%%%%%%%%%%%%%
%%%%%%%%% average - noisy


\subsubsection{Za pomocą mediany}
Wyniki wykonanych operacji prezentują obrazki \ref{fig.lenac_median}, \ref{fig.lena_median}, \ref{fig.lenac_normal3_median}, \ref{fig.lenac_uniform3_median}, \ref{fig.lenac_impulse3_median}, \ref{fig.lena_normal3_median}, \ref{fig.lena_uniform3_median} i \ref{fig.lenac_impulse3_median}.

%%%%%%%%%%%%%%%%%%%%%%%%%%%%%%%%%%%%%%%%%%%%%%%%%%%%%%%%%%%%%%%%%%%%%%%%%%%%%%%%%%%%%%%%%%%%%%%%%%%
%%%%%%%%% median - original

%%%%%%%%%%%%%%%%%%%%%%%%%%%%%%%%%%%%%%%%%%%%%%%%%%%%%%%%%%%%%%%%%%%%%%%%%%%%%%%%%%%%%%%%%%%%%%%%%%%
%%%%%%%%% median - noisy

\subsection{Modyfikacja w oparciu o histogram}
Poniżej przedstawiony został przykład przetwarzania obrazu w oparciu o edycję histogramu – koloru czerwieni:

Do uzyskania przedstawionego obrazu jako minimalna wartość koloru została podana wartość 108, natomiast jako parametr $\alpha$ przyjęta została wartość 54.

Na załączonych histogramach widać, że poszczególne piksele przyjmują wartości koloru czerwonego nie mniejsze niż wartość minimalna. Zmianie również ulega histogram obrazujący odcienie szarości.

Można zauważyć, że jakakolwiek zmiana wartości zarówno parametru minimalnej wartości koloru piksela jak i parametru $\alpha$ powoduje wzrost odcieni czerwieni na obrazie.

\subsection{Filtracja liniowa w oparciu o splot}
Obrazki \ref{fig.lenac_convolution} i \ref{fig.lena_convolution} pokazują wyniki dokonanych splotów.

\subsection{Filtracja nieliniowa}
Obrazki \ref{fig.lenac_rosenfeld} i \ref{fig.lena_rosenfeld} pokazują wyniki dokonanych filtracji.

\subsection{Obiektywna ocena jakości obrazu}
Tabela \ref{tab.lena_noised} pokazuje wartości MSE i SNR dla zaszumionych, niefiltrowanych obrazów.

\begin{table}
  \centering
  \begin{tabular}{|l|c|c|}
    \hline
    \textbf{Obrazek} & \textbf{MSE} & \textbf{SNR} \\
    \hline\hline
    lenac\_normal3.png (\ref{fig.lenac_normal3}) & 3272.54 & 66.7987 \\
    \hline
    lenac\_uniform3.png (\ref{fig.lenac_uniform3}) & 5215.6 & 64.7745 \\
    \hline
    lenac\_impulse3.png (\ref{fig.lenac_impulse3}) & 2600.51 & 67.797 \\
    \hline
    lena\_normal3.png (\ref{fig.lena_normal3}) & 3573.38 & 65.8764 \\
    \hline
    lena\_uniform3.png (\ref{fig.lena_uniform3}) & 3731.03 & 65.6889 \\
    \hline
    lena\_impulse3.png (\ref{fig.lena_impulse3}) & 4820.2 & 64.5765 \\
    \hline
  \end{tabular}
  \caption{Wartości MSE i SNR dla oryginalnych, zaszumionych obrazów}
  \label{tab.lena_noised}
\end{table}

%%%%%
% filtered originals:

% lenac_average_3x3.png & 73.7015 & 83.2728 \\
% lenac_average_5x5.png & 204.175 & 78.8475 \\
% lenac_average_7x7.png & 333.349 & 76.7186 \\
% lenac_median_3x3.png & 47.5571 & 85.1754 \\
% lenac_median_5x5.png & 137.835 & 80.554 \\
% lenac_median_7x7.png & 234.332 & 78.2492 \\
% lena_average_3x3.png & 73.3598 & 82.7526 \\
% lena_average_5x5.png & 204.384 & 78.3027 \\
% lena_average_7x7.png & 336.728 & 76.1344 \\
% lena_median_3x3.png & 46.4018 & 84.7418 \\
% lena_median_5x5.png & 134.196 & 80.1298 \\
% lena_median_7x7.png & 227.615 & 77.8351 \\

\subsubsection{Filtrowanie wykorzystujące średnią arytmetyczną}
Tabelki \ref{tab.lenac_average} i \ref{tab.lena_average} pokazują wartości MSE i SNR dla obrazków przefiltrowanych za pomocą filtru uśredniającego.

\begin{table}
  \centering
  \begin{tabular}{|l|c|c|}
    \hline
    \textbf{Obrazek} & \textbf{MSE} & \textbf{SNR} \\
    \hline\hline
    lenac\_normal3\_average\_3x3.png (\ref{fig.lenac_normal3_average_3x3}) & 460.261 & 75.3175 \\
    \hline
    lenac\_normal3\_average\_5x5.png (\ref{fig.lenac_normal3_average_5x5}) & 362.608 & 76.3532 \\
    \hline
    lenac\_normal3\_average\_7x7.png (\ref{fig.lenac_normal3_average_7x7}) & 430.668 & 75.6061 \\
    \hline
    lenac\_uniform3\_average\_3x3.png (\ref{fig.lenac_uniform3_average_3x3}) & 711.122 & 73.4281 \\
    \hline
    lenac\_uniform3\_average\_5x5.png (\ref{fig.lenac_uniform3_average_5x5}) & 479.162 & 75.1427 \\
    \hline
    lenac\_uniform3\_average\_7x7.png (\ref{fig.lenac_uniform3_average_7x7}) & 511.669 & 74.8577 \\
    \hline
    lenac\_impulse3\_average\_3x3.png (\ref{fig.lenac_impulse3_average_3x3}) & 385.965 & 76.0821 \\
    \hline
    lenac\_impulse3\_average\_5x5.png (\ref{fig.lenac_impulse3_average_5x5}) & 335.781 & 76.687 \\
    \hline
    lenac\_impulse3\_average\_7x7.png (\ref{fig.lenac_impulse3_average_7x7}) & 416.618 & 75.7502 \\
    \hline
  \end{tabular}
  \caption{Wyniki filtru uśredniającego dla obrazków kolorowych}
  \label{tab.lenac_average}
\end{table}

\begin{table}
  \centering
  \begin{tabular}{|l|c|c|}
    \hline
    \textbf{Obrazek} & \textbf{MSE} & \textbf{SNR} \\
    \hline\hline
    lena\_normal3\_average\_3x3.png (\ref{fig.lena_normal3_average_3x3}) & 488.09 & 74.5222 \\
    \hline
    lena\_normal3\_average\_5x5.png (\ref{fig.lena_normal3_average_5x5}) & 368.478 & 75.743 \\
    \hline
    lena\_normal3\_average\_7x7.png (\ref{fig.lena_normal3_average_7x7}) & 431.016 & 75.0622 \\
    \hline
    lena\_uniform3\_average\_3x3.png (\ref{fig.lena_uniform3_average_3x3}) & 500.129 & 74.4163 \\
    \hline
    lena\_uniform3\_average\_5x5.png (\ref{fig.lena_uniform3_average_5x5}) & 369.692 & 75.7288 \\
    \hline
    lena\_uniform3\_average\_7x7.png (\ref{fig.lena_uniform3_average_7x7}) & 431.345 & 75.0589 \\
    \hline
    lena\_impulse3\_average\_3x3.png (\ref{fig.lena_impulse3_average_3x3}) & 674.032 & 73.1204 \\
    \hline
    lena\_impulse3\_average\_5x5.png (\ref{fig.lena_impulse3_average_5x5}) & 475.249 & 74.6379 \\
    \hline
    lena\_impulse3\_average\_7x7.png (\ref{fig.lena_impulse3_average_7x7}) & 519.425 & 74.2519 \\
    \hline
  \end{tabular}
  \caption{Wyniki filtru uśredniającego dla obrazków w skali szarości}
  \label{tab.lena_average}
\end{table}

\subsubsection{Filtrowanie wykorzystujące medianę}
Tabelki \ref{tab.lenac_median} i \ref{tab.lena_median} pokazują wartości MSE i SNR dla obrazków przefiltrowanych za pomocą filtru medianowego.

\begin{table}
  \centering
  \begin{tabular}{|l|c|c|}
    \hline
    \textbf{Obrazek} & \textbf{MSE} & \textbf{SNR} \\
    \hline\hline
    lenac\_normal3\_median\_3x3.png (\ref{fig.lenac_normal3_median_3x3}) & 185.699 & 79.2594 \\
    \hline
    lenac\_normal3\_median\_5x5.png (\ref{fig.lenac_normal3_median_5x5}) & 186.602 & 79.2384 \\
    \hline
    lenac\_normal3\_median\_7x7.png (\ref{fig.lenac_normal3_median_7x7}) & 272.073 & 77.6007 \\
    \hline
    lenac\_uniform3\_median\_3x3.png (\ref{fig.lenac_uniform3_median_3x3}) & 149.399 & 80.2041 \\
    \hline
    lenac\_uniform3\_median\_5x5.png (\ref{fig.lenac_uniform3_median_5x5}) & 180.319 & 79.3871 \\
    \hline
    lenac\_uniform3\_median\_7x7.png (\ref{fig.lenac_uniform3_median_7x7}) & 264.651 & 77.7208 \\
    \hline
    lenac\_impulse3\_median\_3x3.png (\ref{fig.lenac_impulse3_median_3x3}) & 61.0756 & 84.0889 \\
    \hline
    lenac\_impulse3\_median\_5x5.png (\ref{fig.lenac_impulse3_median_5x5}) & 147.514 & 80.2592 \\
    \hline
    lenac\_impulse3\_median\_7x7.png (\ref{fig.lenac_impulse3_median_7x7}) & 241.002 & 78.1273 \\
    \hline
  \end{tabular}
  \caption{Wyniki filtru medianowego dla obrazków kolorowych}
  \label{tab.lenac_median}
\end{table}

\begin{table}
  \centering
  \begin{tabular}{|l|c|c|}
    \hline
    \textbf{Obrazek} & \textbf{MSE} & \textbf{SNR} \\
    \hline\hline
    lena\_normal3\_median\_3x3.png (\ref{fig.lena_normal3_median_3x3}) & 112.754 & 80.8859 \\
    \hline
    lena\_normal3\_median\_5x5.png (\ref{fig.lena_normal3_median_5x5}) & 169.632 & 79.1121 \\
    \hline
    lena\_normal3\_median\_7x7.png (\ref{fig.lena_normal3_median_7x7}) & 254.932 & 77.3429 \\
    \hline
    lena\_uniform3\_median\_3x3.png (\ref{fig.lena_uniform3_median_3x3}) & 182.968 & 78.7834 \\
    \hline
    lena\_uniform3\_median\_5x5.png (\ref{fig.lena_uniform3_median_5x5}) & 187.981 & 78.666 \\
    \hline
    lena\_uniform3\_median\_7x7.png (\ref{fig.lena_uniform3_median_7x7}) & 267.995 & 77.1259 \\
    \hline
    lena\_impulse3\_median\_3x3.png (\ref{fig.lena_impulse3_median_3x3}) & 68.9829 & 83.0197 \\
    \hline
    lena\_impulse3\_median\_5x5.png (\ref{fig.lena_impulse3_median_5x5}) & 149.28 & 79.6671 \\
    \hline
    lena\_impulse3\_median\_7x7.png (\ref{fig.lena_impulse3_median_7x7}) & 239.461 & 77.6148 \\
    \hline
  \end{tabular}
  \caption{Wyniki filtru medianowego dla obrazków w skali szarości}
  \label{tab.lena_median}
\end{table}

\section{Dyskusja}
Spośród zastosowanych operacji najmniejsze znaczenie ma negatyw. Nie poprawia on postrzegania obrazu przez ludzkie oko, nie uwydatnia także szczegółów.

Zarówno w przypadku rozjaśniania jak i zwiększania kontrastu wydobywane są pewne szczegóły z tła \ppauza jak na przykład włosy lub frędzle z kapelusza. W przypadku rozjaśniania jednak cały obraz staje się zamglony. Zwiększenie kontrastu powoduje zwiększenie czytelności obrazka poprzez uwydatnienie różnic w pomiędzy znajdującymi się obok różnymi barwami.

Przyciemnianie obrazka powoduje pogorszenie jego odbioru \ppauza zacierane są różnice nie tylko pomiędzy szczegółami, ale także obiektami na pierwszym planie.

Zmniejszenie kontrastu powoduje częściowe ,,wyszarzenie'' obrazka, pozostawia jednak pierwszy plan dobrze widocznym.

Chociaż zarówno w przypadku zmiany jasności jak i zmiany kontrastu wartości były tak samo oddalone od zera, to odbiór tak zmienionych obrazów jest różny \ppauza znacznie lepiej i wyraźniej postrzegane są obrazy zmodyfikowane dodatnimi wartościami. Wynikać to może z budowy ludzkiego oka, które słąbiej reaguje na ciemne barwy.

Przedstawione zaszumione obrazy charakteryzują się dużymi wartościami MSE oraz relatywnie małym współczynnikiem SNR. Widoczny jest na nich dość dobrze pierwszy plan, choć wprowadzony szum znacznie utrudnia odróżnienie szczegółów. Poddanie obrazów procesowi filtracji polepsza odbiór obrazków, zależny jest on jednak w dużej mierze od rodzaju szumu i wybranej metody filtracji.

Filtrowanie za pomocą średniej arytmetycznej znacznie zmniejsza średni błąd kwadratowy, jednak wyniki działania tego filtru akceptowalne są tylko w~przypadku maski o rozmiarze $3 \times 3$ \ppauza przy większych wielkościach obraz staje się bardzo rozmazany. Ten rodzaj filtru sprawdza się najlepiej w przypadku szumu o rozkładzie normalnym \ppauza w pozostałych przypadkach pozostawia zbyt wiele śladów szumu. Ponadto, wadą tego filtru jest wprowadzanie do obrazu nowych wartości pikseli, co powoduje zanik ostrości elementów.

Filtr medianowy daje znacznie lepsze efekty od filtru opartego o średnią arytmetyczną. Dla maski $3 \times 3$, w szczególności dla szumu o rozkładzie impulsowym, osiąga zadowalające rezultaty pozwalając jednocześnie na zachowanie względnie dużej ilości szczegółów. Zwiększenie wielkości maski powoduje usunięcie większości pozostałych artefaktów kosztem drobnego pogorszenia widzialności szczegółów.

Filtrowanie liniowe z opisanymi wyżej maskami powoduje uwypuklenie krawędzi w zadanych kierunkach. Filtr taki może nadać się do łatwiejszego rozpoznania obiektów, gdyż usuwana jest znaczna część informacji dotycząca koloru, pozostawiane są natomiast zmiany w danym kierunku.

Filtr nieliniowy z operatorem Rosenfelda również uwidacznia \ppauza w pewnym stopniu \ppauza kontury, jednak wyniki dawane przez ten filtr powodują u autorów zawroty głowy i ciężko jest odnaleźć zastosowanie dla tej metody.

\subsection{Efektywna implementacja splotu}
W ogólnym przadku splot dla maski o wymiarach $[N, M]$ wymaga $N \cdot M$ opracji mnożenia, $N \cdot M - 1$ opracji dodawania oraz jednego dzielenia. W przypadku gdy rozmiary i wartości maski znane są zawczasu, można zaoszczędzić część operacji.

W naszym przypadku każda z masek posiada pięć wartości równych 1, trzy -1 oraz jedną -2. Łatwo zauważyć, że elementów o współczynnikach wynoszących 1 nie potrzeba mnożyć, gdyż wynik pozostanie bez zmian. Ponadto, zamiast mnożyć wartości pikseli przez -1, można zastosować unarny operator minusa \ppauza można przypuszczać, że zostanie ona potraktowana przez kompilator jako chęć pobrania wartości przeciwnej, co prawdopodobnie zostanie rozwinięte do rozkazu procesora \verb|neg| zamiast \verb|mul| \ppauza istnieje duża szansa, że ten pierwszy wykona się szybciej.

Dzięki opisanemu powyżej sposobowi, zredukowaliśmy ilość koniecznych kosztownych mnożeń z 9 do zaledwie jednego. Ilość operacji dodawania i~dzielenia pozostaje bez zmian.

\begin{thebibliography}{99}
\bibitem{zrodlo1}
Christian Graus, \textit{Image Processing for Dummies}, \url{http://www.codeproject.com/KB/GDI-plus/csharpgraphicfilters11.aspx}, [online, dostęp 12.04.2011]

\bibitem{zrodlo2}
Tomasz Lubiński, Dariusz Rorat, \textit{Przetwarzanie obrazów}, \url{http://www.algorytm.org/przetwarzanie-obrazow/}, [online, dostęp 12.04.2011]
\end{thebibliography}

\end{document}