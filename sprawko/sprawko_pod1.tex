\documentclass{classrep}
\usepackage[utf8]{inputenc}
\usepackage[pdftex]{graphicx}
\usepackage[polish]{babel}
\usepackage{algorithm}
\usepackage{algorithmic}
\usepackage{multicol}
\usepackage{amsmath}
\usepackage{listings}
\usepackage{array}
\usepackage{multirow}
\usepackage{url}

\studycycle{Informatyka, studia dzienne, II st.}
\coursesemester{I}

\coursename{Przetwarzanie obrazów i dźwięków}
\courseyear{2010/2011}

\courseteacher{xxxxxx}
\coursegroup{xxxxxx}
\svnurl{xxxxxx}

\author{%
  \studentinfo{Michał Janiszewski}{169485} \and
  \studentinfo{Michał Kawski}{xxxxxx}
}

\title{Zadanie 1: Przetwarzanie obrazów}

\floatname{algorithm}{Algorytm}

\begin{document}

\maketitle

\section{Cel zadania}
Celem zadania było stworzenie szkieletu aplikacji do przetwarzania i analizy obrazów i dźwięków oraz zaimplementowanie następujących operacji dotyczących przetwarzania obrazu:
\begin{itemize}
 \item operacje podstawowe:
  \begin{itemize}
   \item zmiana kontrastu obrazu,
   \item zmiana jasności obrazu,
   \item wyznaczanie negatywu obrazu,
  \end{itemize}
 \item filtrowanie obrazu w oparciu o:
  \begin{itemize}
   \item filtr ze średnią arytmetyczną,
   \item filtr medianowy,
  \end{itemize}
 \item wyznaczanie i pokazywanie histogramu obrazu dla poszczególnych jego kanałów, a także umożliwienie edycji obrazu na podstawie tego histogramu,
 \item filtracja liniowa oparta o splot z możliwością zdefiniowania rozmiaru i wartości maski,
 \item filtracja nieliniowa w dziedzinie czasu.
\end{itemize}

Wylosowane warianty zadań to:
\begin{itemize}
 \item Histogramy H3 \ppauza wyjściowa gęstość prawdopodobieństwa podana wzorem Raleigha:
  \begin{equation*}
   g(f) = g_{min} + \left(2 \alpha^2 \ln \left(\frac{1}{N} \displaystyle\sum\limits_{m=0}^f H\left(m\right)\right)^{-1}\right)^{\frac{1}{2}}
  \end{equation*}
 \item Splot S4 \ppauza wydobywanie szczegółów z tła: południe, południowy-zachód, zachód, północny-zachód:
\begin{equation*}
\left(\begin{array}{ccc} -1 & -1 & -1 \\ 
1 & -2 & 1 \\
1 & 1 & 1 \\
\end{array}\right)\\
\\
\left(\begin{array}{ccc} 1 & -1 & -1 \\ 
1 & -2 & -1 \\
1 & 1 & 1 \\
\end{array}\right)\\
\\
\left(\begin{array}{ccc} 1 & 1 & -1 \\ 
1 & -2 & -1 \\
1 & 1 & -1 \\
\end{array}\right)\\
\\
\left(\begin{array}{ccc} 1 & 1 & 1 \\ 
1 & -2 & -1 \\
1 & -1 & -1 \\
\end{array}\right)
\end{equation*}
  \item filtracja nieliniowa O5 \ppauza operator Rosenfelda:
\begin{equation*}
 g(x, y) = \frac{1}{R} \left( \displaystyle \sum \limits_{i=1}^R f\left(x + i - 1, y\right) - \displaystyle \sum \limits_{i=1}^R f \left(x - i, y\right) \right)
\end{equation*}

  
\end{itemize}


\section{Metoda rozwiązania}



\begin{thebibliography}{99}
%\bibitem{pleple}
% opis
\end{thebibliography}

\end{document}
